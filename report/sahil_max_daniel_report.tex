\documentclass[letterpaper,twocolumn,10pt]{article}
\usepackage{usenix2019_v3}

% to be able to draw some self-contained figs
\usepackage{tikz}
\usepackage{amsmath}

% inlined bib file
\usepackage{filecontents}

%-------------------------------------------------------------------------------
\begin{filecontents}{\jobname.bib}
%-------------------------------------------------------------------------------

@misc{ghadiry_2017, title={Is My Google Home Spying On Me?}, url={https://labs.sogeti.com/google-home-spying/}, journal={SogetiLabs}, author={Ghadiry, Amir}, year={2017}, month={Apr}}

@misc{morgenstern_morgenstern_2018, title={Careless Whisper: Does Amazon Echo send data in silent mode?}, url={https://www.iot-tests.org/2017/06/careless-whisper-does-amazon-echo-send-data-in-silent-mode/}, journal={AV-TEST Internet of Things Security Testing Blog}, author={Morgenstern, Maik and Morgenstern, Maik}, year={2018}, month={Jul}}

@misc{market_reports, title={Global Voice Assistant Market Analysis \& Forecast 2017 To 2023}, url={https://www.marketreportsworld.com/global-voice-assistant-market-analysis-forecast-2017-to-2023-10850832}, journal={market Reports}, year={2017}, month={Jul}}

@inproceedings{Diao2014YVA2666620.2666623,
 author = {Diao, Wenrui and Liu, Xiangyu and Zhou, Zhe and Zhang, Kehuan},
 title = {Your Voice Assistant is Mine: How to Abuse Speakers to Steal Information and Control Your Phone},
 booktitle = {Proceedings of the 4th ACM Workshop on Security and Privacy in Smartphones \&\#38; Mobile Devices},
 series = {SPSM '14},
 year = {2014},
 isbn = {978-1-4503-3155-5},
 location = {Scottsdale, Arizona, USA},
 pages = {63--74},
 numpages = {12},
 url = {http://doi.acm.org/10.1145/2666620.2666623},
 doi = {10.1145/2666620.2666623},
 acmid = {2666623},
 publisher = {ACM},
 address = {New York, NY, USA},
 keywords = {android security, permission bypassing, speaker, voice assistant, zero permission attack},
} 



\end{filecontents}

%-------------------------------------------------------------------------------
\begin{document}
%-------------------------------------------------------------------------------

%don't want date printed
\date{}

% make title bold and 14 pt font (Latex default is non-bold, 16 pt)
\title{\Large \bf IoTShark: Monitoring and Analyzing IoT Traffic}

%for single author (just remove % characters)
\author{
{\rm Sahil Gandhi, Max Wang, Daniel Achee}\\
University of California, Los Angeles \\
\{sahilmgandhi,yingbowang, dpachee\}@ucla.edu
} % end author

\maketitle

%-------------------------------------------------------------------------------
\begin{abstract}
%-------------------------------------------------------------------------------
IoT devices continue to flood stores and homes at alarming rates with no indication of slowing down. 
They do provide helpful functions from controlling lights, blinds, networks and more, but at a great cost. 
Both the devices and the services are controlled by the same company (often times) and thus users have no idea what exactly is being transmitted. 
There are tools to set up man-in-the-middle attacks and sniff the network but they require some non-negligible time to set up and require extensive knowledge about computer security. 
We present IoTShark as a bootstrapping tool to quickly let users hook onto an IoT device on their network and monitor what is being sent. 
They can store these "traces" as csv files for long term use and can also quickly view graphs and statistics of their traces in a GUI to observe is any extra or anomalous network traffic is being generated by the devices.

\end{abstract}

%-------------------------------------------------------------------------------
\section{Introduction}
%-------------------------------------------------------------------------------

In recent years there has been a proliferation of voice assistants in people’s homes. 
These useful IoT devices provide a convenient voice interface for users, but also present some privacy concerns as they are an always-on listening device. 
Companies selling these products, such as Amazon, claim that the devices are constantly listening for a keyword, such as "Alexa", and only after hearing this keyword, record and process what users say. 
Therefore, anything a user says before a keyword is issued and after their voice transaction has occurred is not recorded and private. 
Unfortunately, there is no easy way for a user to verify that the device is upholding public statements and the privacy agreement. 
Companies do admit that what a user says following this keyword is used for targeting advertisements and customizing content. 
This creates a conflict of interest as there is a financial incentive for companies to collect more voice data on their users and abuse their agreements.

These devices are relatively new and have emerged due to advances in natural language processing. 
These advances have allowed for fairly accurate keyword detection at the edge device in real time without the need to send the data to the cloud. 
With the increased accuracy of natural language processing and the ability to pull more semantic meaning from phrases, these devices are able to provide services that make them attractive to consumers and advertising agencies alike.

%-------------------------------------------------------------------------------
\section{Background and Motivations}
%-------------------------------------------------------------------------------

There are many players in the market, with the most popular being Amazon Alexa, Google Home Assistant, Apple Siri, and Microsoft Cortana. 
There is no standardization amongst voice assistants, with each using different protocols and traffic patterns. 
These devices typically encrypt the traffic. This encryption is a double edged sword as it both protects a user from attackers sniffing their data, but also prevents the user from auditing the behavior of the device. 

Given that voice assistants appear to be only expanding in popularity and are projected to be a 7840.82 million dollar market by 2023 \cite{market_reports}, it is imperative that users, even those that are not technically sophisticated, can easily audit and understand the behavior of their devices.

%-------------------------------------------------------------------------------
\section{Related Work}
%-------------------------------------------------------------------------------

Given that the data being transmitted across IoT devices is encrypted (end-to-end) it is quite difficult to discern what exactly is being transmitted, and even complicated man-in-the-middle style attacks can’t be used since the data payload format and security cert \slash auth format are unknown to a random attacker. 
Despite these hindrances, there has been some work in the field by bloggers and researchers to capture the information being transmitted and infer some patterns. 

Blog posts such as \cite{ghadiry_2017} and \cite{morgenstern_morgenstern_2018} go into great detail for setting up man-in-the-middle style proxies to capture all data being sent across the network by an IoT device (a Google Home and an Amazon Echo respectively for these blogs) and use an application such as Wireshark to sniff packets and determine whether they originated from an IoT device. 
We use and build off of these techniques to capture data from our own IoT devices during our experiments and for creating our deliverables mentioned below. 
However, these blog posts are not always easy to follow without extensive computer science knowledge and are often times outdated. We aim to create a plug and play tool that takes this complexity away from the user.

TODO: Include something about TLS-RAR here? and another IoT Paper?

Lastly, as more modules ("skills" in the Alexa world) are created, our smart assistants become ever smarter and are able to do more, include such sensitive tasks like writing emails, calling other individuals, and more. 
The following paper, \cite{Diao2014YVA2666620.2666623}, determines the relative ease with which a hacker can mimic a user and send off malicious requests. 
Even someone without malicious intents can take advantage of the information and skills of a voice assistant and breach a user’s privacy. 
Since 2014, when the previous paper was published, Google, Amazon and others have taken the initiative to alert users after their device was accessed (and send a transcript \slash recording of the voice data) and protect some sensitive information. 
However the casual user is not aware of the repercussions of purchasing/using the device. 
% We hope that our monitoring tool can serve as this "wake up" call that the smart assistant is a very versatile device that should be treated with the same level of privacy and scrutiny that we treat our laptops \slash phones and monitoring what it is doing should be easy enough for the common non-technical person to set up on their own.

%-------------------------------------------------------------------------------
\section{System Overview}
%-------------------------------------------------------------------------------

\subsection{Design}
%-----------------------------------

%-------------------------------------------------------------------------------
\section{Evaluation}
%-------------------------------------------------------------------------------

%-------------------------------------------------------------------------------
\section{Discussion and Future Work}
%-------------------------------------------------------------------------------

\subsection{Discussion}
%-----------------------------------

\subsection{Future Work}
%-----------------------------------

An interesting paper in this space is which aims to classify events in different IoT streams, in particular focusing on classifying when a Nest Camera is live streaming data vs motion detecting, when an Alexa is transmitting data back to the server, and a couple other applications. We intended to do this classification but quickly realized that we would need to collect vast amounts of data from sources that are not just the three of us. However as a future task, we intend to build off of the work from this paper to detect and classify IoT traffic, initially for smart assistants and later for other devices. 

%-------------------------------------------------------------------------------
\section{Conclusion}
%-------------------------------------------------------------------------------

%-------------------------------------------------------------------------------
\bibliographystyle{plain}
\nocite{*}
\bibliography{\jobname}

%%%%%%%%%%%%%%%%%%%%%%%%%%%%%%%%%%%%%%%%%%%%%%%%%%%%%%%%%%%%%%%%%%%%%%%%%%%%%%%%
\end{document}
%%%%%%%%%%%%%%%%%%%%%%%%%%%%%%%%%%%%%%%%%%%%%%%%%%%%%%%%%%%%%%%%%%%%%%%%%%%%%%%%

%%  LocalWords:  endnotes includegraphics fread ptr nobj noindent
%%  LocalWords:  pdflatex acks